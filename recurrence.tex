\documentclass{article}
\usepackage{amsmath}
\title{Recurrence}
\author{Prof. Micha Hofri}
\newtheorem{gsr}{Theorem}
\newcommand{\twopartdef}[4]
{
  \left\{
  \begin{array}{ll}
    #1 & \mbox{} #2 \\
    #3 & \mbox{} #4
  \end{array}
  \right.
}
\begin{document}
\maketitle
First order linear recurrence:\\
\[
x_{n+1}=b_nx_n+a_n\qquad n\geq m
\]
in which ${a_n}$, ${b_n}$ and $x_m$ are known.\\
Then we have the following general form to calculate $X_{n+1}$\\
\begin{align*}
x_{n+1}&=\sum_{j=m}^na_j\prod_{i=j+1}^nb_i+x_m\prod_{i=m}^nb_i \\
&=a_n(\sum_{j=m}^{n-1}a_j\prod_{i=j+1}^{n-1}b_i+x_m\prod_{i=m}^{n-1}b_i)+b_n \\
\end{align*}
Using the above formulation, let's look at an interesting example:\\
Assuming a loan mortagage instance, which has the following definitions:\\
\begin{center}
$A$ - loan amount \\
$m$ - monthly payment \\
$\rho$ - monthly interest rate \\
$R_n$ - principal at the end of month n \\
$R_0=A$ \\
$M=360$ - loan duration \\
\end{center}
Each month the remaining money we have to pay is:
\[
R_{n+1}=R_n(1+\rho)-m
\]
Then we have $R_M=0$ because we have no money to pay when we have paid
$M$ months and paid off the loan. And now we want to calculate the
monthly payment m.\\
\begin{align*}
  R_M&=\sum_{j=0}^{M-1}(-m)\prod_{i=j+1}^{M-1}(1+\rho)+A(1+\rho)^M\\
  &=\sum_{j=0}^{M-1}(-m)(1+\rho)^{M-j-1}+A(1+\rho)^M\\
  &=(1+\rho)^M(A-m\sum_{j=0}^{M-1}\frac{1}{(1+\rho)^M}\\
  &=(1+\rho)^MA-\frac{m}{\rho}((1+\rho)^M-1)=0
\end{align*}
So
\[
m=\frac{A\cdot\rho(1+\rho)^M}{(1+\rho)^M-1}
\]
When we have the following instance:
\begin{align*}
A&=100,000\\
\rho&=\frac{0.06}{12}=0.005
\end{align*}
Then we have $m=598.98$. \\

Next, let's continue. First, give some calculation. \\
\[
\binom{n}{m}=\binom{n-1}{m}+\binom{n-1}{m-1}=\frac{n}{m}\binom{n-1}{m-1}
\]
Proof:\\
\begin{align}
\binom{n}{m}&=\frac{n!}{m!(n-m)!} \\
\binom{n-1}{m-1}&=\frac{(n-1)!}{(n-m)!(m-1)!} \\
\binom{n-1}{m}&=\frac{(n-1)!}{(n-m-1)!m!}
\end{align}
(2)$\cdot \frac{n}{m}$ apparently equals to (1)\\
Define $S_n$ the summation of the first n terms of $f(n)$, so \\
\[
S_n=\sum_{k=1}^nf(k)=\sum_{k=1}^{n-1}f(k)+f(n)
\]
Then if we are given:\\
\[
a_n=\sum_{k=1}^n\binom{n}{k}\frac{(-1)^{k+1}}{k}
\]
what can we derive from the above equation?\\
Using the above conclusion:\\
\begin{align*}
a_n&=\sum_{k=1}^n\left(\binom{n-1}{k-1}\frac{(-1)^{k+1}}{k}+\binom{n-1}{k}\frac{(-1)^{k+1}}{k}\right)
  \\
&=\underbrace{\sum_{k=1}^{n-1}\binom{n-1}{k}\frac{(-1)^{k+1}}{k}}_{a_{n-1}}+\sum_{k=1}^n\frac{(-1)^{k+1}}{n}\binom{n}{k}\\
&=a_{n-1}-\frac{1}{n}\underbrace{\sum_{k=1}^n\binom{n}{k}\cdot(-1)^k}_{\text{binomial
  distribution}} \\
&=a_{n-1}+\frac{1}{n}
\end{align*}
Then we know that:\\
\[
a_n = \sum_{k=1}^{n-1}b_k+b_n\qquad b_n = \frac{1}{n}
\]
Then we get $a_n=H_n$.\\
 \\
Compositions\\
Look at the following problem:\\
\[
n=a_1+a_2+...+a_k\qquad a_k\geq 0,a_k\in N,n\in N
\]
In how many ways can we obtain an exact above equation?\\
We define $C_{n,k}$ as the number of ways obtaining the above equation
using k numbers.\\
Let's reformulate the above problem. It is same as the problem that we
have $n$ $1s$, using bars to partition the $1s$ into $k$
segements. Obviously $k-1$ bars are needed. \\
Let's reformulate this problem again, it can be represented as we have
$n+k-1$ element, choosing  $k-1$ to change to bars.\\
Then we have:\\
I can't remember detail of the following  part\\
----------------------------------------------------------------------------------------\\
\begin{align*}
C_{n.k}&=\sum_{i=0}^nC_{n-i,k-1}\\
&=\sum_{j=0}^nC_{j,k-1}\\
&=\sum_{j=0}^{n-1}C_{j,k-1}+C_{n,k-1}\\
&=C_{n-1,k-1}+C_{n,k-1}
\end{align*}
\begin{align*}
C_{n,k}^{(1)}&\overset{?}{=}\binom{n+k}{n}\overset{!}{=}\binom{n+k}{k}\\
C_{n,k}^{(2)}&\overset{?}{=}\binom{n+k-1}{k-1}=\binom{n+k-3}{k-2}+\binom{n+k-2}{k-2}
\end{align*}
-------------------------------------------------------------------------------------------\\
General Solution of Recurrence\\
May contain parameters, that can be used to fit initial values.\\
$x_n^{(P)}$ - Particular Solution pays no attention to initial values\\
$x_n^{(0)}$ - Homogeneous Recurrence Replaces $b_n$ by $0$.
\begin{gsr}
The general solution of a recurrence can be written as the sum of a
general solution of the homogeneous recurrence and a particular
solution of the fully recurrence.(Init values may not be satisfied),
which can be formulated as the following representation:\\
\[
x_n=x_n^{(0)}+x_n^{(P)}
\]
\end{gsr}
Example:\\
\[
x_{n+1}=2x_n+1\qquad n\geq0,x_0=2
\]
Hom.:\\
\[
x_{n+1}=2x_n=A\cdot 2^n
\]
Par.:(Simply letting $x_{n+1}=x_n$)\\
\[
x_n=-1
\]
Then:\\
\begin{align*}
x_n&=A\cdot 2^n-1\\
2&=A\cdot 2^0-1\\
A&=3
\end{align*}
Hence:\\
\[
x_n=3\cdot 2^n-1
\]
A more complicated example:\\
we have $n$ coins, define a independent random variable \\
\[
C_i=\twopartdef{H}{\text{with prob }\frac{1}{2i+1}}{T}{\text{with prob }1-\frac{1}{2i+1}=\frac{2i}{2i+1}}
\]
and \\
\begin{center}
$A_n$={tosses of all $n$ coins giving an odd number of $Hs$}\\
\end{center}
\begin{align*}
P(A_n)&=P_n \\
&=Pr(C_n(H))\cdot (1-P_{n-1})+Pr(C_n(T))\cdot P_{n-1}\\
&=\frac{1}{2n+1}(1-P_{n-1}+\frac{2n}{2n+1}P_{n-1}\\
&=P_{n-1}(\frac{2n}{2n+1}-\frac{1}{2n+1})+\frac{1}{2n+1} \qquad n\geq 1,P_0=0 
\end{align*}
Hom.:\\
\[
P_n^{(0)}=\frac{2n-1}{2n+1}P_{n-1}^{(0)}=\frac{3}{2n+1}k
\]
Par.:\\
\[
g=g\frac{2n-1}{2n+1}+\frac{1}{2n+1} \\
g=\frac{1}{2}
\]
Hence:\\
\begin{align*}
P_n&=\frac{3k}{2n+1}+\frac{1}{2} \\
P_0&=\frac{3k}{1}+\frac{1}{2}=0\\
k&=\frac{1}{6} \\
P_n&=\frac{1}{2}-\frac{1}{4n+2} \\
&=\frac{1}{2}(1-\frac{1}{2n+1}) \\
&=\frac{n}{2n+1}
\end{align*}

\end{document}

%%% Local Variables: 
%%% mode: latex
%%% TeX-master: t
%%% End: 
