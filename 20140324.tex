\documentclass[11pt]{article}
\usepackage{times,mathptmx}
\usepackage{amsmath,amsthm,amssymb}
\usepackage{xspace}
\newcommand{\DS}{\displaystyle}
\newcommand{\pmfa}{{\sc pmf}\xspace}
\newcommand{\pmf}{ probability mass function\xspace}
\newcommand{\pgfa}{{\sc pgf}\xspace}
\newcommand{\pgf}{ probability generating function\xspace}
\newcommand{\ogfa}{{\sc ogf}\xspace}
\newcommand{\ogf}{ ordinary generating function\xspace}
\newcommand{\egfa}{{\sc egf}\xspace}
\newcommand{\mgf}{ moment generating function\xspace}
\newcommand{\egf}{ exponential generating function\xspace}
\newcommand{\pdf}{ probability density function\xspace}
\newcommand{\gf}{generating function\xspace}
\newcommand{\gfa}{{\sc gf}\xspace}
\newcommand{\bc}{binomial coefficient\xspace}
\newcommand{\pfd}{ partial fraction decomposition\xspace}
\newcommand{\lhs}{left-hand side\xspace}
\newcommand{\rhs}{right-hand side\xspace}
\newcommand{\lf}{\lfloor}
\newcommand{\rf}{\rfloor}
\newcommand{\lc}{\lceil}
\newcommand{\rc}{\rceil}
\newcommand{\lp}{\left(}
\newcommand{\rp}{\right)}
\newcommand{\lbr}{\left[}
\newcommand{\rbr}{\right]}
\newcommand{\ul}{\underline}
\newcommand{\ovl}{\overline}
\newcommand{\cA}{{\mathcal A}} \newcommand{\cB}{{\mathcal B}}
\newcommand{\cC}{{\mathcal C}} \newcommand{\cD}{{\mathcal D}}
\setlength{\parindent}{0pt}
\setlength{\parskip}{1ex}
\newtheorem{DEFINITION}{Definition}
\newenvironment{df}{\begin{DEFINITION} \hspace{.30em}  \rm}%
          {\hfill\protect$\lhd$\end{DEFINITION}}
\newtheorem{EXAMPLE}{Example}
\newenvironment{example}{\begin{EXAMPLE} \hspace{.30em} \rm}%
                      {\end{EXAMPLE}}
\author{Prof. Micha Hofri/Typeset:  Chao Li, Xinyue Liu}
\title{CS504 3/24/2014 notes}
\newcommand{\twopartdef}[4]
{
  \left\{
    \begin{array}{ll}
      #1 & \mbox{if } #2 \\
      #3 & \mbox{} #4 
    \end{array}
  \right.
}

\begin{document}
\maketitle

\section{Asymptotic Analysis}

\subsection{Types}
\begin{enumerate}
\item logorithmic 
\item polynomial: $n^a (a>0) $ $a$ has not to be integer
\item exponential: $b^n (b>1)$ 
\item super-exponential $ $
\end{enumerate}

\subsection{}
Let's look at a baby example, calculate an approximation of function \(f_n=\frac{2}{n+2}\) and function \(f_n=\frac{n}{n+2}\)\\
\[ f_n = \frac{2}{n+2} = \frac{n+2-2}{n+2} = 1 - \frac{2}{n+2} = 1+ O(\frac{1}{n}) \]
\[
f_n=\frac{n^2}{n^2+2}=\frac{n^2+2-2}{n^2+2}=1-\frac{2}{n^2+2}=1+O(\frac{1}{n^2})
\]
\subsection{}
\[ n^a \in o(n^b) \rightarrow 0<a<b \]

$ C_n = \frac{n^a}{n^b} = \frac{1}{n^{b-a}}$, $C_n$ converges to $0$ when $b>a$ as $n$ increases.

\[a^n \in o(b^n) \rightarrow 1<a<b \]

$ C_n = \frac{a^n}{b^n} = (\frac{a}{b})^n$, $C_n$ decreases as $n$ increases when $b>a>1$.

\subsection{}
Next, if we have \(n^a\in o(b^n)\), in which \(a>0, b^n>1\), can we conclude that \( C_n = \frac{n^a}{b^n} \rightarrow 0 \)? \\

\[\frac{C_{n+1}}{C_n} = \frac{(n+1)^a}{n^a} \cdot \frac{b^n}{b^{n+1}} = (1+\frac1n)^a \frac1b\]\\
if we want $\frac{C_{n+1}}{C_n} < 1$, then:
\\
\[(1+\frac1n) < b \rightarrow log(1+\frac1n) < \frac{logb}{a} \rightarrow \frac1n < \frac{logb}{a} \] \\
\\
Thus, when $n>\frac{a}{logb}$, we have $C_n \rightarrow 0$


\section{Asymptotic Manipulations}

\subsection{Asymptotic scale}

Assume we have serires: $g_0(n), g_1(n), g_2(n) ...$, and we know $g_i(n) \in \Omega [g_{i+1}(n)]$ 
\\
Define \\
\[f_n = \sum_{i=0}^k g_i{n} + O(g_{k+1}^n)\]
, which is \textit{Poincaré series}. \\
Then the generating function of \emph{Poincar Series} is \\
\[f(r)=f_0+f_1r+f_2r^2+...\]
 which coverges in some $ r<A\qquad (A>1)$.Thus,

\begin{align*}
f(r)  &= \underbrace{f_0 + f_1r + ... +f_kr^k }_{\sum_{k\geq0} k!}+ r^{k+1}(\underbrace{f_{k+1}+ f_{k+2} + ....}_{\mbox{finite value}}) \\
      &= f_0 + f_1r + ... + f_kr^k + O(r^{k+1})
\end{align*}

\subsection{}
Now consider \\
\[ f(n) = n^{\frac{a}{n}} \]
perform some derivation \\
\[
f(n)=\frac{a}{n}=e^{lnn^{\frac{a}{n}}}=e^{\frac{a}{n}lnn}
\]
\[
n^{\frac{a}{n}}=1+a\frac{lnn}{n}+O\left((\frac{lnn}{n})^2\right)
\]
\[
\frac{n^{\frac{a}{n}}-1-a\frac{lnn}{n}}{(\frac{lnn}{n})^2}\rightarrow 2
\]
\section{Exponential Integral}
Now let's look at some integral functions, first \\
\[
f(x)=\int_{x}^{\infty}\frac{e^{x-t}}{t}dt
\]
Perform some calculation\\
\begin{align*}
f(x)&=\left.-\frac{e^{x-t}}{t}\right|_x^\infty-\int_x^\infty\frac{e^{x-t}}{t^2}dt \\
&=\frac{1}{x}+\left.\frac{e^{x-t}}{t^2}\right|_x^\infty+2\int_x^\infty\frac{e^{x-t}}{t^3} \\
&=\frac{1}{x}-\frac{1}{x^2}+\frac{2!}{x^3}-\frac{3!}{x^4}\pm ...+\frac{(-1)^{k-1}(k-1)!}{x^k}+(-1)^kk!\int_x^\infty\frac{e^{x-t}}{t^{k+1}}dt
\end{align*}
Let  \(S_k(x)=\frac{1}{x}-\frac{1}{x^2}+\frac{2!}{x^3}-\frac{3!}{x^4}\pm ...+\frac{(-1)^{k-1}(k-1)!}{x^k}\), then \\
\[
f(x)=S_k(x)+(-1)^kk!\int_x^\infty \frac{e^{x-t}}{t^{k+1}}dt
\]
Then\\
\begin{align*}
|f(x)-S_k(x)|&=k!\int_x^\infty\frac{e^{x-t}}{t^{k+1}}dt \\
&\ll k!\int_x^\infty\frac{1}{t^{k+1}}dt \\
&=\frac{k!}{k+2}\cdot \frac{1}{x^{k+2}}
\end{align*}
\subsection{}
Next example, calculate the approximation of \(\frac{n}{n-1}\)
\[
\frac{n}{n-1}=\frac{1}{1-\frac{1}{n}}
\]
As \(\frac{1}{1-\frac{1}{n}}=\sum_{l\geq 0}(\frac{1}{n})^l\)
\[
\frac{n}{n-1}=1+\frac{1}{n}+O(\frac{1}{n^2})
\]
Next example \\
\[
\frac{n}{n-1}ln\frac{n}{n-1}
\]
calculate the approximation of the above function with a asymptotic of \(O(\frac{1}{n^4})\)
First, calculate part of the function \\
\[
ln\frac{n}{n-1}=ln\frac{1}{1-\frac{1}{n}}=-ln(1-\frac{1}{n})=\frac{1}{n}+\frac{1}{2n^2}+\frac{1}{3n^3}+O(\frac{1}{n^4})
\]
Then the whole function is \\
\[
\frac{n}{n-1}ln\frac{n}{n-1}=\left(1+\frac{1}{n}+\frac{1}{n^2}+\frac{1}{n^3}+O\left(\frac{1}{n^4}\right)\right)\left(\frac{1}{n}+\frac{1}{2n^2}+\frac{1}{3n^3}+O\left(\frac{1}{n^4}\right)\right)
\]
\[
\begin{matrix}
=\frac{1}{n} & + &\frac{1}{2n^2} & + & \frac{1}{3n^3} & + & ... \\
         ~       & ~ &\frac{1}{n^2}  & + & \frac{1}{2n^3} & + & ... \\
         ~       & ~ &       ~           & ~ &\frac{1}{n^3} & + & ...
\end{matrix}
\]
\[
=\frac{1}{n}+\frac{3}{2n^2}+\frac{11}{6}\frac{1}{n^3}+O\left(\frac{1}{n^4}\right)
\]
Next example \\
\[
e^{0.1}+cos(0.1)-ln(0.9)
\]
calculate the above function with a asymptotic approximation of \(O(10^4)\) \\
\[
e^x=1+x+\frac{x^2}{2}+\frac{x^3}{6}+\frac{x^4}{24}+\cdots
\]
\[
cos(x)=1-\frac{x^2}{2}+\frac{x^4}{24}+\cdots
\]
\[
ln(0.9)=ln(1-x)|_{0.1}=-x-\frac{x^2}{2}-\frac{x^3}{3}-\frac{x^4}{4}\cdots
\]
\[
e^{0.1}+cos(0.1)-ln(0.9)
\]
\[
\begin{matrix}
= & 1 & + & 0.1 & + & 1/2\cdot 1/100 & + & 1/6\cdot 1/1000 & +\cdots \\
~ & 1 & ~ &  ~  & - & 1/2\cdot 1/100 & + & \cdots & ~ \\
~ & ~ & - & 0.1 & - & 1/2\cdot 1/100 & - & 1/3\cdot 1/1000 & +\cdots \\
= & 2 & - & 1/2\cdot 1/100 &-& 1/6\cdot 1/1000
\end{matrix}
\]

\end{document}

