\documentclass[11pt]{article}
\usepackage{times,mathptmx}
\usepackage{amsmath,amsthm,amssymb}
\usepackage{xspace}
\newcommand{\DS}{\displaystyle}
\newcommand{\pmfa}{{\sc pmf}\xspace}
\newcommand{\pmf}{ probability mass function\xspace}
\newcommand{\pgfa}{{\sc pgf}\xspace}
\newcommand{\pgf}{ probability generating function\xspace}
\newcommand{\ogfa}{{\sc ogf}\xspace}
\newcommand{\ogf}{ ordinary generating function\xspace}
\newcommand{\egfa}{{\sc egf}\xspace}
\newcommand{\mgf}{ moment generating function\xspace}
\newcommand{\egf}{ exponential generating function\xspace}
\newcommand{\pdf}{ probability density function\xspace}
\newcommand{\gf}{generating function\xspace}
\newcommand{\gfa}{{\sc gf}\xspace}
\newcommand{\bc}{binomial coefficient\xspace}
\newcommand{\pfd}{ partial fraction decomposition\xspace}
\newcommand{\lhs}{left-hand side\xspace}
\newcommand{\rhs}{right-hand side\xspace}
\newcommand{\lf}{\lfloor}
\newcommand{\rf}{\rfloor}
\newcommand{\lc}{\lceil}
\newcommand{\rc}{\rceil}
\newcommand{\lp}{\left(}
\newcommand{\rp}{\right)}
\newcommand{\lbr}{\left[}
\newcommand{\rbr}{\right]}
\newcommand{\ul}{\underline}
\newcommand{\ovl}{\overline}
\newcommand{\RR}{\mathbb{R}} \newcommand{\TT}{\mathbb{T}}
\newcommand{\ZZ}{\mathbb{Z}} \newcommand{\QQ}{\mathbb{Q}}
\newcommand{\cA}{{\mathcal A}} \newcommand{\cB}{{\mathcal B}}
\newcommand{\cC}{{\mathcal C}} \newcommand{\cD}{{\mathcal D}}
\setlength{\parindent}{0pt}
\setlength{\parskip}{1ex}
\newtheorem{DEFINITION}{Definition}
\newenvironment{df}{\begin{DEFINITION} \hspace{.30em}  \rm}%
          {\hfill\protect$\lhd$\end{DEFINITION}}
\newtheorem{EXAMPLE}{Example}
\newenvironment{example}{\begin{EXAMPLE} \hspace{.30em} \rm}%
                      {\end{EXAMPLE}}
\author{Prof. Micha Hofri/Typeset:  Chao Li, Xinyue Liu}
\title{CS504\ \  3/24/2014 notes}
\newcommand{\twopartdef}[4]
{
  \left\{
    \begin{array}{ll}
      #1 & \mbox{if } #2 \\
      #3 & \mbox{} #4 
    \end{array}
  \right.
}

\begin{document}
\maketitle

\section{Manipulation of asymptotic Expansions}

\subsection{Main asymptotic classes:}
The following are the main classes we encounter in analysis of algorithms
and data structures.  Remember that asymptotic class is related to rate of
growth, not actual size.
\begin{enumerate}
\item logarithmic: $~\log n$
\item polynomial: $~n^a (a\in \RR_+$; that is, a strictly positive real
number).
\item exponential: $~b^n (b>1)$ 
\item super-exponential (many types, {\em e.g.} $3^{n^2},\, 2^{2^n}$
\end{enumerate}

As usual when dealing with asymptotics, functions are assigned to a class
based on the leading term --- $2^{3n} + n^3$ is an exponential function.
We also disregard numerical multiplying factors, $n^2$ and $1000n^2+70n$
are both quadratic polynomials.

\subsection{Simple examples, absolute and relative error terms}
Let  \(f_n=\frac{2}{n+2}\) and \(g_n=\frac{n^2}{n^2-5}\) be functions we want
to emphasize their change as $n$ increases.  That is the purpose of
asymptotic expansions.
\begin{align*}
f_n = \frac{2}{n+2} =\frac2n \frac1{1+\frac2n}
= \frac 2n \lp 1 - \frac2n + \frac 4{n^2} \mp \cdots \rp
= \frac2n -\frac4{n^2} +O\lp\frac{1}{n^3}\rp
\end{align*}
The last term is an absolute error term, of order $O\lp\frac{1}{n^3}\rp$,
and since the leading term is of order $1/n$, the error term is a relative
error of order $O\lp n^{-2}\rp$.
\begin{align*}
g_n=\frac{n^2}{n^2-5}=\frac1{1-\frac5{n^2}}
=1 +\frac5{n^2} +\frac{25}{n^4} +\cdots 
=1 +\frac5{n^2} \lp 1+O\lp\frac{1}{n^2}\rp\rp
\end{align*}
Here the last term appears to be a $n^{-2}$ relative error term, relative to
the 1 it shares a pair of parentheses with, but this is not the right
point of view: it needs to be compared with the leading term in the
expression, which is 1 as well, hence it is both an absolute error term
of order $n^{-4}$, and also such a relative error term.

\subsection{Polynomials are ordered by their degree}
That means
\[
f_n = n^a, \quad g_n=n^b, \text{ and } 0<a<b \implies f_n \in o(g_n).
\]
And the reason is is that
$C_n = f_n/g_n =n^{a-b}$, converges to $0$ when $b>a$ as $n$ increases.

While the claim made above was for positive powers, the same algebra holds
for negative powers: the functions decrease, and at a faster rate as the
powers go down; thus $n^{-3} \in o\lp n^{-1.7}\rp$.  The same ordering:
smaller is ``smaller.''

Similarly, exponentials are ordered by their base:
\[
u_n = c^n, \quad v_n=d^n, \text{ and } 0<c<d \implies u_n \in o(v_n),
%a^n \in o(b^n) \rightarrow 1<a<b
\]
for the same reason,
$ C_n =u_n/v_n =c^n/d^n = (c/d)^n\to 0$, as $n$ increases since $(c/d)<1$.

\subsection{Polynomials are slower than exponentials}
Now we compare functions from different asymptotic classes.  Proving that
logarithms are slower to increase than polynomials was given as homework
assignment.

Next, we show the claim in the title of the subsection.  Let
$f_n = n^a, \quad g_n=b^n$ and the numerical parameters satisfy
$a>0$ and $b>1$.  The proof method is the same: we form the ratio
$C_n = f_n/g_n$, and show it converges to zero.  We do this by creating
one more ratio, $C_{n+1}/C_n$, and show that at some value of $n$ (which can
be very small or very large, it does not matter for the validity of the
claim, which is very patient), the ratio becomes smaller than 1, and
stays so indefinitely as $n$ increases.

\[
\frac{C_{n+1}}{C_n} = \frac{(n+1)^a}{n^a} \cdot \frac{b^n}{b^{n+1}}
= \lp 1+\frac1n\rp^a \frac1b = \frac1b+\frac a{bn}\lp1+o(1)\rp.
\]
The $o(1)$ appears since the next term is of lower order than $1/n$.
The fraction $1/b$ is smaller than 1, and once $\DS \frac a{bn}<1-\frac1b$ is
satisfied, at $n_o=a/(b-1)$, we know that for all $n> n_1=2n_o$ the value
of the last ratio will stay under 1, and therefore $C_n\to0$.
The doubling of $n_o$
was made to clear away whatever debris the $o(1)$ above can bring.


% if we want $\frac{C_{n+1}}{C_n} < 1$, then:
% \\
% \[(1+\frac1n) < b \rightarrow log(1+\frac1n) < \frac{logb}{a} \rightarrow \frac1n < \frac{logb}{a} \] \\
% \\
% Thus, when $n>\frac{a}{logb}$, we have $C_n \rightarrow 0$


\section{Asymptotic Manipulations}

\subsection{Asymptotic scale}

Assume we have serires: $g_0(n), g_1(n), g_2(n) ...$, and we know $g_i(n) \in \Omega [g_{i+1}(n)]$ 
\\
Define \\
\[f_n = \sum_{i=0}^k g_i{n} + O(g_{k+1}^n)\]
, which is \textit{Poincaré series}. \\
Then the generating function of \emph{Poincar\'e Series} is \\
\[f(r)=f_0+f_1r+f_2r^2+...\]
 which coverges in some $ r<A\qquad (A>1)$.Thus,

\begin{align*}
f(r)  &= \underbrace{f_0 + f_1r + ... +f_kr^k }_{\sum_{k\geq0} k!}+ r^{k+1}(\underbrace{f_{k+1}+ f_{k+2} + ....}_{\mbox{finite value}}) \\
      &= f_0 + f_1r + ... + f_kr^k + O(r^{k+1})
\end{align*}

\subsection{}
Now consider \\
\[ f(n) = n^{\frac{a}{n}} \]
perform some derivation \\
\[
f(n)=\frac{a}{n}=e^{lnn^{\frac{a}{n}}}=e^{\frac{a}{n}lnn}
\]
\[
n^{\frac{a}{n}}=1+a\frac{lnn}{n}+O\left((\frac{lnn}{n})^2\right)
\]
\[
\frac{n^{\frac{a}{n}}-1-a\frac{lnn}{n}}{(\frac{lnn}{n})^2}\rightarrow 2
\]
\section{Exponential Integral}
Now let's look at some integral functions, first \\
\[
f(x)=\int_{x}^{\infty}\frac{e^{x-t}}{t}dt
\]
Perform some calculation\\
\begin{align*}
f(x)&=\left.-\frac{e^{x-t}}{t}\right|_x^\infty-\int_x^\infty\frac{e^{x-t}}{t^2}dt \\
&=\frac{1}{x}+\left.\frac{e^{x-t}}{t^2}\right|_x^\infty+2\int_x^\infty\frac{e^{x-t}}{t^3} \\
&=\frac{1}{x}-\frac{1}{x^2}+\frac{2!}{x^3}-\frac{3!}{x^4}\pm ...+\frac{(-1)^{k-1}(k-1)!}{x^k}+(-1)^kk!\int_x^\infty\frac{e^{x-t}}{t^{k+1}}dt
\end{align*}
Let  \(S_k(x)=\frac{1}{x}-\frac{1}{x^2}+\frac{2!}{x^3}-\frac{3!}{x^4}\pm ...+\frac{(-1)^{k-1}(k-1)!}{x^k}\), then \\
\[
f(x)=S_k(x)+(-1)^kk!\int_x^\infty \frac{e^{x-t}}{t^{k+1}}dt
\]
Then\\
\begin{align*}
|f(x)-S_k(x)|&=k!\int_x^\infty\frac{e^{x-t}}{t^{k+1}}dt \\
&\ll k!\int_x^\infty\frac{1}{t^{k+1}}dt \\
&=\frac{k!}{k+2}\cdot \frac{1}{x^{k+2}}
\end{align*}
\subsection{}
Next example, calculate the approximation of \(\frac{n}{n-1}\)
\[
\frac{n}{n-1}=\frac{1}{1-\frac{1}{n}}
\]
As \(\frac{1}{1-\frac{1}{n}}=\sum_{l\geq 0}(\frac{1}{n})^l\)
\[
\frac{n}{n-1}=1+\frac{1}{n}+O(\frac{1}{n^2})
\]
Next example \\
\[
\frac{n}{n-1}ln\frac{n}{n-1}
\]
calculate the approximation of the above function with a asymptotic of
\(O(\frac{1}{n^4})\) \\
First, calculate part of the function \\
\[
ln\frac{n}{n-1}=ln\frac{1}{1-\frac{1}{n}}=-ln(1-\frac{1}{n})=\frac{1}{n}+\frac{1}{2n^2}+\frac{1}{3n^3}+O(\frac{1}{n^4})
\]
Then the whole function is \\
\[
\frac{n}{n-1}ln\frac{n}{n-1}=\left(1+\frac{1}{n}+\frac{1}{n^2}+\frac{1}{n^3}+O\left(\frac{1}{n^4}\right)\right)\left(\frac{1}{n}+\frac{1}{2n^2}+\frac{1}{3n^3}+O\left(\frac{1}{n^4}\right)\right)
\]
\[
\begin{matrix}
=\frac{1}{n} & + &\frac{1}{2n^2} & + & \frac{1}{3n^3} & + & ... \\
         ~       & ~ &\frac{1}{n^2}  & + & \frac{1}{2n^3} & + & ... \\
         ~       & ~ &       ~           & ~ &\frac{1}{n^3} & + & ...
\end{matrix}
\]
\[
=\frac{1}{n}+\frac{3}{2n^2}+\frac{11}{6}\frac{1}{n^3}+O\left(\frac{1}{n^4}\right)
\]
Next example \\
\[
e^{0.1}+cos(0.1)-ln(0.9)
\]
calculate the above function with a asymptotic approximation of \(O(10^4)\) \\
\[
e^x=1+x+\frac{x^2}{2}+\frac{x^3}{6}+\frac{x^4}{24}+\cdots
\]
\[
cos(x)=1-\frac{x^2}{2}+\frac{x^4}{24}+\cdots
\]
\[
ln(0.9)=ln(1-x)|_{0.1}=-x-\frac{x^2}{2}-\frac{x^3}{3}-\frac{x^4}{4}\cdots
\]
\[
e^{0.1}+cos(0.1)-ln(0.9)
\]
\[
\begin{matrix}
= & 1 & + & 0.1 & + & 1/2\cdot 1/100 & + & 1/6\cdot 1/1000 & +\cdots \\
~ & 1 & ~ &  ~  & - & 1/2\cdot 1/100 & + & \cdots & ~ \\
~ & ~ & - & 0.1 & - & 1/2\cdot 1/100 & - & 1/3\cdot 1/1000 & +\cdots \\
= & 2 & - & 1/2\cdot 1/100 &-& 1/6\cdot 1/1000
\end{matrix}
\]
Now comes our final example of this part\\
As it is known that\\
\begin{align*}
H_n&=lnn+r+\frac{a}{n}+O(\frac{1}{n^2}) \\
H_{n+1}&=ln(n+1)+r+\frac{a}{n+1}
\end{align*}

Then we have\\
\[
H_{n+1}-H_n=\frac{1}{n+1}=ln\frac{n+1}{n}+a(\frac{1}{n+1}-\frac{1}{n})+?
\]
what is \(a\) \\
From the above equation, and
\(ln\frac{n+1}{n}=ln(1+\frac{1}{n})=\frac{1}{n}-\frac{1}{2n^2}\) we can get
\begin{align*}
(\frac{1}{n+1}-\frac{1}{n})(1-a)+\frac{1}{n}&=\frac{1}{n}-\frac{1}{2n^2}+\cdots
\\
(a-1)\frac{1}{n(n+1)}&=-\frac{1}{2n^2}
\end{align*}
Since
\(\frac{1}{n(n+1)}=\frac{1}{n^2}\left(\frac{1}{1+\frac{1}{n}}\right)\approx\frac{1}{n^2}\),
then \\
\[
a-1=-\frac{1}{2}
\]
\[
a=\frac{1}{2}
\]
So \\
\[
H_n=lnn+r+\frac{1}{2n}+?\qquad ?\text{ is }\frac{1}{12n^2}
\]
\end{document}

\begin{align*}
\end{align*}
