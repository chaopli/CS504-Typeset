\documentclass[11pt]{article}
\usepackage{times,mathptmx}
\usepackage{amsmath,amsthm,amssymb}
\usepackage{xspace}
\newcommand{\DS}{\displaystyle}
\newcommand{\pmfa}{{\sc pmf}\xspace}
\newcommand{\pmf}{ probability mass function\xspace}
\newcommand{\pgfa}{{\sc pgf}\xspace}
\newcommand{\pgf}{ probability generating function\xspace}
\newcommand{\ogfa}{{\sc ogf}\xspace}
\newcommand{\ogf}{ ordinary generating function\xspace}
\newcommand{\egfa}{{\sc egf}\xspace}
\newcommand{\mgf}{ moment generating function\xspace}
\newcommand{\egf}{ exponential generating function\xspace}
\newcommand{\pdf}{ probability density function\xspace}
\newcommand{\gf}{generating function\xspace}
\newcommand{\gfa}{{\sc gf}\xspace}
\newcommand{\bc}{binomial coefficient\xspace}
\newcommand{\pfd}{ partial fraction decomposition\xspace}
\newcommand{\lhs}{left-hand side\xspace}
\newcommand{\rhs}{right-hand side\xspace}
\newcommand{\lf}{\lfloor}
\newcommand{\rf}{\rfloor}
\newcommand{\lc}{\lceil}
\newcommand{\rc}{\rceil}
\newcommand{\lp}{\left(}
\newcommand{\rp}{\right)}
\newcommand{\lbr}{\left[}
\newcommand{\rbr}{\right]}
\newcommand{\ul}{\underline}
\newcommand{\ovl}{\overline}
\newcommand{\RR}{\mathbb{R}} \newcommand{\TT}{\mathbb{T}}
\newcommand{\ZZ}{\mathbb{Z}} \newcommand{\QQ}{\mathbb{Q}}
\newcommand{\cA}{{\mathcal A}} \newcommand{\cB}{{\mathcal B}}
\newcommand{\cC}{{\mathcal C}} \newcommand{\cD}{{\mathcal D}}
\setlength{\parindent}{0pt}
\setlength{\parskip}{1ex}
\newtheorem{DEFINITION}{Definition}
\newenvironment{df}{\begin{DEFINITION} \hspace{.30em}  \rm}%
          {\hfill\protect$\lhd$\end{DEFINITION}}
\newtheorem{EXAMPLE}{Example}
\newenvironment{example}{\begin{EXAMPLE} \hspace{.30em} \rm}%
                      {\end{EXAMPLE}}
\author{Prof. Micha Hofri/Typeset:  Chao Li}
\title{Generating Functions - Part 2}
\begin{document}
\maketitle
\section*{Negating Upper Argument}
For binomial coefficient, we have the following formula:\\
\[
\binom{a}{b}=\binom{-a+b-1}{b}(-1)^b
\]
Then let's consider the following example\\
let \(\DS a_n=\binom{n}{m}\), for \(n\geq m\), then we have:\\
\[
a_n=\binom{n}{m}=\binom{n}{n-m}
\]
and its generating function:
\[
a(x)=\sum_{n\geq m}x^n\binom{n}{m}=\sum_{n\geq m}x^n\binom{n}{n-m}
\]
substitute our former formula into this equation, we then get
\[
a(x)=\sum_{n\geq m}x^n\binom{-n+n-m-1}{n-m}(-1)^{n-m}
\]
Let \(n-m=j\) then we have 
\[
a(x)=\sum_{j\geq 0}x^{j-m}\binom{-n+n-m-1}{j}(-1)^j=x^m\sum_{j\geq 0}\binom{-m-1}{j}(-x)^j=\frac{x^m}{(1-x)^{m+1}}
\]
\section*{An Example}
Assume we have \(\DS C_n=1+(-1)^n, n\geq 0\)
then its generating function is 
\begin{align*}
C(x) &= \sum_{n\geq 0}(1+(-1)^n)x^n \\
	 &= \sum_{n\geq 0}x^n+\sum_{n\geq 0}(-x)^n \\
	 &= \frac{1}{1-x}+\frac{1}{1+x}=\frac{2}{1-x^2} \\
	 &= \sum_{k\geq 0}2(x^2)^k
\end{align*}
\section*{Scaling}
Assume we have two series \(\{a_n\}\), \(\{b_n\}\), and a constant \(c\). We have \(\DS b_n\overset{\text{def}}{=}c^na_n\)
So the generating function of them could have the following feature that
\[
b(x)=\sum(xc)^na_n=a(cx)
\]
\end{document}