\documentclass[11pt]{article}
\usepackage{times,mathptmx}
\usepackage{amsmath,amsthm,amssymb}
\usepackage{xspace}
\newcommand{\DS}{\displaystyle}
\newcommand{\pmfa}{{\sc pmf}\xspace}
\newcommand{\pmf}{ probability mass function\xspace}
\newcommand{\pgfa}{{\sc pgf}\xspace}
\newcommand{\pgf}{ probability generating function\xspace}
\newcommand{\ogfa}{{\sc ogf}\xspace}
\newcommand{\ogf}{ ordinary generating function\xspace}
\newcommand{\egfa}{{\sc egf}\xspace}
\newcommand{\mgf}{ moment generating function\xspace}
\newcommand{\egf}{ exponential generating function\xspace}
\newcommand{\pdf}{ probability density function\xspace}
\newcommand{\gf}{generating function\xspace}
\newcommand{\gfa}{{\sc gf}\xspace}
\newcommand{\bc}{binomial coefficient\xspace}
\newcommand{\pfd}{ partial fraction decomposition\xspace}
\newcommand{\lhs}{left-hand side\xspace}
\newcommand{\rhs}{right-hand side\xspace}
\newcommand{\lf}{\lfloor}
\newcommand{\rf}{\rfloor}
\newcommand{\lc}{\lceil}
\newcommand{\rc}{\rceil}
\newcommand{\lp}{\left(}
\newcommand{\rp}{\right)}
\newcommand{\lbr}{\left[}
\newcommand{\rbr}{\right]}
\newcommand{\ul}{\underline}
\newcommand{\ovl}{\overline}
\newcommand{\cA}{{\mathcal A}} \newcommand{\cB}{{\mathcal B}}
\newcommand{\cC}{{\mathcal C}} \newcommand{\cD}{{\mathcal D}}
\setlength{\parindent}{0pt}
\setlength{\parskip}{1ex}
\newtheorem{DEFINITION}{Definition}
\newenvironment{df}{\begin{DEFINITION} \hspace{.30em}  \rm}%
          {\hfill\protect$\lhd$\end{DEFINITION}}
\newtheorem{EXAMPLE}{Example}
\newenvironment{example}{\begin{EXAMPLE} \hspace{.30em} \rm}%
                      {\end{EXAMPLE}}
\author{Prof. Micha Hofri/Typeset:  Chao Li, Xinyue Liu}
\title{CS504 3/20/2014 notes}
\newcommand{\twopartdef}[4]
{
  \left\{
    \begin{array}{ll}
      #1 & \mbox{if } #2 \\
      #3 & \mbox{} #4 
    \end{array}
  \right.
}
\begin{document}
\maketitle
\section{On derangements and singletons in permutations}
First let us look at some definitions:\\
\begin{df}
A {\bf cycle} of size $k$ in a permutation of $[n]$, the first $n$ natural
numbers, is a sequence of $k$ elements $(a_1, a_2, \dots, a_k)$ which are 
in positions $(a_2, a_3, \dots, a_k, a_1)$.

{\bf Singletons, Matches, Coincidences} are to items in a
permutation which are in their original positions. The term
singletons is short for a cycle of a single item, or of length one.

{\bf Derangement} is a permutation which has no matches.

$D_n$ is the number of derangements in $S_n$, the set of all permutations
of $[n]$ (also called the {\bf symmetric group}).

$d_{n,k}$ -- the number of permutations in $S_n$ where $D_n=k$.
% \item[$\overline{D_n}$] -$d_{n,0}$, means the permutation is fully deranged.
\end{df}
\begin{example}\label{Mp4.1}
In the permutation
$\{4,2,1,3\}$. $(1,3,4)$ is a 3-cycle and $(2)$ is a 1-cycle. Hence
this permutation is among the $d_{4,1}$ such permutations in $S_4$.  How
many such permutations exist? We shall answer it below.
\end{example}
From the above definitions we have
\[
d_{n,k}=\binom{n}{k}\,D_{n-k},
\]
which means that once  we choose $k$ positions for singletons, which we
can do in $\binom nk$ ways, we have $D_{n-k}$ ways to arrange entries in
the 
other $n-k$ positions, so that they are all `deranged.'\\
We have a couple of ways to approach the calculation of the number of
derangements, and we do it via the following recurrence, due to L.~Euler:
\[
D_n=(n-1)(D_{n-1}+D_{n-2}), \qquad n\geq 2,~ D_0=1,~D_1=0
\]
{\em Query:\ } Can you explain the reasons for the two initial values?

The intent is to solve the recurrence  using the \egfa, and we then shift the
indices in the recurrence, so they cover the entire range of definition of
the sequence, and find
\[
D_{n+2}=(n+1)(D_{n+1}+D_{n}), \qquad n\ge 0,~ D_0=1,~D_1=0
\]
The \egfa of $D_n$ is
\[
\hat{D}(x)=\sum_{n\ge 0}D_n\,\frac{x^n}{n!}.
\]
We shall use the standard transformation rules for \egfa{s}, where
we use the notation $\cD_x$ (or for brevity, just $\cD$) for the
differentiation operator with respect to $x$:\\
$a_{n+k} \mapsto \cD^k\hat a(x)$, and
$n^k a_{n} \mapsto (x\cD)^k\hat a(x)$,\\
which also gives $(n+1) a_{n+1} \mapsto \cD x\cD\hat a(x)$.\\
Then, 
\begin{align*}
\hat{D}''(x)&=\left(x\hat{D}'(x)\right)'+x\hat{D}'(x)+\hat{D}(x)\\
&=x\hat{D}''(x)+D'(x)+x\hat{D}'(x)+\hat{D}(x).\quad \text{Collecting }
\hat D''(x),\\
\hat{D}''(x)(1-x)&=\hat{D}(x)+(1+x)\hat{D}'(x).
\end{align*}
Observe that $((1-x)\hat D(x))'' = (1-x)\hat D''(x) -2\hat D'(x)$, and use
it to rewrite the last \lhs in the equation above, so that we find
\[
\left((1-x)\hat{D}(x)\right)'' +2\hat D'(x) =\hat{D}(x)+(1+x)\hat{D}'(x),
\]
and rearranging,
\begin{align*}
\left((1-x)\hat{D}(x)\right)'' &= \hat{D}(x)+(x-1)\hat{D}'(x))
= \left((x-1)\hat{D}(x)\right)'\\
& =-\left((1-x)\hat{D}(x)\right)'
\end{align*}
Let $f(x)=(1-x)\hat{D}(x)$, then we have
\begin{align*}
f''(x)&=-f'(x) \implies \frac{f''}{f'}=-1, \qquad \text{A logarithmic
derivative}\\
\ln f'&=c-x~ \text{which leads to}~ f'=e^{c-x}=ke^{-x}\\
  f(x)&=k_1+k_2e^{-x} \qquad \text{A second integration adds a constant}
\end{align*}
\begin{align*}
\text{Hence}\quad
  \hat{D}(x)=\frac{k_1+k_2e^{-x}}{1-x}, \xrightarrow{D_0=1}k_1+k_2=1 \\
  D_1=\hat{D}'(0)\xrightarrow{D_1=0} 
  \left.\frac{(-k_2e^{-x}(1-x)+(k_1+k_2e^{-x})}{1}\right|_{x=0} 
\end{align*}
Then we get \(k_1=0, k_2=1\), then we know that
$\hat{D}(x)=\frac{e^{-x}}{1-x}$.  The expansion of the exponential
function is $e^{-x} =\sum_{r\geq 0}\frac{(-x)^r}{r!}$, and having it
divided by $(1-x)$ means that $[x^n]\hat D(x)$ is the $n$-prefix of this
expansion, that is
\begin{align*}
  D_n=n!\sum_{k=0}^n\frac{(-1)^k}{k!} = n!e_n(-x) \approx \frac{n!}{e}
\end{align*}
The expression $e_n(-x)$ is called \textbf{Incomplete Exponential
Function}.

Next define $M_n$ as the number of matches in a permutation of size $n$.
We now know its distribution:
\[
Pr[M_n=k] = \frac{d_{n,k}}{n!}.
\]
While this allows us to calculate, for example, all moments of $M_n$, the
first moments is much easier to compute, as is often the case by using
indicator random variables.  Let
\[
Y_j=\twopartdef{1}{j\text{ is in position }j}{0}{\text{o.w.}}
\]
Then we have   $M_n =\sum_{j=1}^n Y_j.$
Now observe that the entry $j$ in the permutation is equally likely to be
in any of the $n$ positions, and in particular, also in its native
position, $j$. 
Therefore
\[
E(M_n)=E\left(\sum_{j=1}^nY_j\right)=
\sum_{j=1}^n\underbrace{E(Y_j)}_{\DS\frac{1}{n}}=1.
\]
\subsection{Solving for $\hat D(x)$ with an inverse relation}
Assume we have two number sequences $\{f_k\}_{k\geq 0}$ and, $\{g_n\}_{n\geq
0}$; we also know the following relation between the two sequences
\[
g_n=\sum_k\binom{n}{k}f_k.
\]
Compute the generating function of \(g_n\)
\begin{align*}
g(x)&=\sum_ng_nx^n \\
  &=\sum_nx^n\sum_k\binom{n}{k}f_k \\
  &=\sum_kf_k\sum_{n\geq 0}\binom{n}{k}x^n \\
  &=\sum_kf_k\frac{x^k}{(1-x)^{k+1}} \\
 &=\frac{1}{1-x}\sum_kf_k\left(\frac{x}{1-x}\right)^k\\
  g(x)&=\frac{1}{1-x}f\left(\frac{x}{1-x}\right) \\
\end{align*}
then let \(t=\frac{x}{1-x}\), and solve for $x$:
\begin{align*}
  x&=t(1-x) \\
  x&=\frac{t}{1+x}\\
\text{also}\quad  \frac{1}{1-x}&=1+t\\
  g\left(\frac{t}{1+t}\right)&=(1+t)f(t).
\end{align*}
Now we have
\begin{align*}
  f_n&=[t^n]f(t) \\
  &=[t^n]\frac{1}{1+t}\sum_{k\geq 0}g_k\frac{t^k}{(1+t)^k} \\
  &=\sum_{k\geq 0}g_k[t^n]\frac{t^k}{(1+t)^{k+1}} \\
  &=\sum_{k\geq 0}g_k[t^{n-k}](1+t)^{-(k+1)}
\end{align*}
As \(\binom{-k-1}{n-k}=\binom{k+1+n-k-1}{n-k}(-1)^{n-k}\) then \\
\[
f_n=\sum_{k\geq 0}g_k(-1)^{n-k}\binom{n}{k}.
\]
As we know, \(d_{n,k}=\binom{n}{k}D_{n-k}\), so
\begin{align*}
n! = \sum_{k=0}^nd_{n,k} =\sum_{k=0}^n\binom{n}{k}D_{n-k} \quad
\xrightarrow{j \equiv n-k} \quad =\sum_{j=0}^n\binom{n}{j}D_j.
\end{align*}
Using the inverse relation we proved above, we can now express the
sequence $\{D_n\}$ in terms of the factorials:
\begin{align*}
D_k&=\sum_j\binom{k}{j}j!(-1)^{k-j}, ~\text{and} \\
\frac{D_k}{k!}
&=\sum_{j=0}^k\frac{(-1)^{k-j}}{ (k-j)!}=\sum_{r=0}^k\frac{(-1)^r}{r!}.
\end{align*}
\end{document}
%%% Local Variables: 
%%% mode: latex
%%% TeX-master: t
%%% End: 
