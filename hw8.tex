\documentclass{article}
\usepackage{amsmath}
\title{HW8}
\author{Chao Li}
\begin{document}
\maketitle
\begin{enumerate}
\item \hfill\\
Let \(p_r=Pr\{X=r\}=pq^{r-1}\), then
\begin{align*}
  P(u)&=\sum_{r\geq 1}p_ru^r \\
  &=\sum_{r\geq 1}pq^{r-1}u^{r-1} \\
  &=up\sum_{r\geq 1}q^{r-1}u^{r-1} \\
  &=\frac{up}{1-uq} \\
\end{align*}
As the definition \\
\[
E[X^n]=\sum_{r\geq 0}p_rr^n
\]
is the n-th moment of the origin, so the 1st moment is \(n=1\) and the
2nd moment is \(n=2\), so
\begin{align*}
E[X^1]&=\sum_{r\geq 0}p_rr=P'(1)\\
&=\frac{p}{(1-q)^2} \\
&=\frac{1}{p}
\end{align*}
\begin{align*}
E[X^2]&=P''(1) \\
&=\frac{-2(1-q)(-q)p}{(1-q)^4} \\
&=\frac{2pq}{(1-q)^3} \\
&=\frac{2q}{p^2}
\end{align*}
Finally,
\begin{align*}
V(X)&=E[X^2]+E[X]-E^2[X] \\
&=\frac{2q}{p^2}+\frac{1}{p}-\frac{1}{p^2} \\
&=\frac{2q+p-1}{p^2}
\end{align*}
\item \hfill\\
a. \\
As we know that \(P[x>k]=1-P[x\leq k]\), then
\begin{align*}
P[x>0]z^0&=z^0(1-P_0) \\
P[x>1]z^1&=z^1(1-p_0-p_1) \\
P[x>2]z^2&=z^2(1-p_0-p_1-p_2) \\
P[x>k]z^k&=z^k(1-\sum_{i=0}^{k}p_k) 
\end{align*}
Sum up the above equations, we get
\begin{align*}
\sum_{k\geq 0}P[x>k]z^k&=\sum_{i\geq 0}z^i-P_0\sum_{i\geq 0}z^i-P_1\sum_{i\geq 1}z^i-...-P_i\sum_{i\geq k}z^i \\
&=(\sum_{k\geq 0}z^k)(1-p_0-p_1z-p_2z^2-p_kz^k) \\
&=\frac{1}{1-z}(1-\sum_{k\geq 0}P_kz^k) \\
&=\frac{1-p(z)}{1-z}
\end{align*}
b. \\
The same as problem 2.a, we have
\begin{align*}
P[x\geq 0]z^0&=z^0(1) \\
P[x\geq 1]z^1&=z^1(1-p_0) \\
P[x\geq 2]z^2&=z^2(1-p_0-p_1) \\
P[x\geq k]z^k&=z^k(1-\sum_{i=0}^{k-1}p_k) 
\end{align*}
Sum up the above equations, we get
\begin{align*}
\sum_{k\geq 0}P[x>k]z^k&=\sum_{i\geq 0}z^i-P_0\sum_{i\geq 1}z^i-P_1\sum_{i\geq 2}z^i-...-P_i\sum_{i\geq k}z^{i+1} \\
&=(\sum_{k\geq 0}z^k)(1-p_0-p_1z-p_2z^2-p_kz^k) \\
&=\frac{1}{1-z}(1-z\sum_{k\geq 0}P_kz^k) \\
&=\frac{1-zp(z)}{1-z}
\end{align*}
\end{enumerate}


\end{document}

%%% Local Variables: 
%%% mode: latex
%%% TeX-master: t
%%% End: 
