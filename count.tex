\documentclass{article}
\usepackage{amsmath}
\title{Count}
\author{Prof. Micha Hofri}
\newcommand{\twopartdef}[4]
{
  \left\{
  \begin{array}{ll}
    #1 & \mbox{if } #2 \\
    #3 & \mbox{if } #4
  \end{array}
  \right.
}
\begin{document}
\maketitle
Let's look at the following pseudocode and count the executing times
of each step:\\
\begin{center}
\begin{tabular}{c c l}
     & count          & \\
1.   & $1$              & $Max(A,n)$ \\
2.   & $1$              & $int\ m, k = 1;$ \\
3.   & $1+X_n$        & $m = A[k]$ \\
4.   & $n$            & $k = k++$ \\
5.   & $n$            & $if(k > n) \overset{1}{\longrightarrow} 8$ \\
6.   & $n-1$          & $if(m>=A[k]) \overset{n-X_n-1}{\longrightarrow} 4$ \\
7.   & $X_n$          & $\overset{X_n}{\longrightarrow} 3$ \\
8.   & $1$              & $return\ m$ \\
\end{tabular}
\end{center}
The code above represents the process of finding the maximum number in
array A with the size of n. The count numbers are indicated, and in
which $X_n$ represents an unknown number whose max value is $n-1$ and
minimun value is $0$. \\
Now we are interested in the average value of $X_n$. We state an
indicator random value: \\
\begin{center}
$y_i = \twopartdef {0}{$no updating of $m$ when $k=i} {1} {m$ is updated
when $k=i}$
\end{center}
In here, $y_i$ is an independent random variable, so we have:\\
\begin{center}
\begin{align}
X_n &= \sum_{i=2}^{n}y_i\\
E[X_n] &= \sum_{i=2}^nE[y_i]=\sum_{i=2}^n\frac{1}{i}=H_n-1
\end{align}
\end{center}
in which, \[E[y_i]=1\cdot{Pr(y_i=1)}=\frac{1}{i}\]\\
\end{document}

%%% Local Variables: 
%%% mode: latex
%%% TeX-master: t
%%% End: 
