\documentclass{article}
\usepackage[lined, boxed, commentsnumbered]{algorithm2e}
\usepackage{amsmath}
\title{Quick Sort}
\author{Prof. Micha Hofri}
\newcommand{\twopartdef}[4]
{
  \left\{
  \begin{array}{ll}
    #1 & \mbox{if } #2 \\
    #3 & \mbox{if } #4
  \end{array}
  \right.
}
\begin{document}
\maketitle
First let's take a look at a version of the Quick Sort Algorithm:\\
\begin{algorithm}[H]
\KwData{$A, i, j$}
QS($A, i, j$)\;
\If{$i>j$} {return\;}
k=partition($A, i, j$)\;
QS($A, i, k-1$)\;
QS($A, k+1, j$)\;
\caption{One Version of Quick Sort Algorithm}
\end{algorithm}
In the algorithm, input $A$ is an array with random permutation of
size n. All the entries are with distinctive value.\\
Apparently, the major part of the $QS$ algorithm is the partition
part, now let's look at a version of the partition algorithm.\\
\begin{algorithm}[H]
partition($A, lo, hi$)\;
p = A[hi]\;
i = lo-1\;
\For{i=lo \emph{\KwTo} hi} {
\If{A[j] $<=$ p}{
i++\;
swap(A[i],A[j])\;
}
}

swap(A[i+1], A[hi])\;
\Return{i+1}
\caption{A Version of Partition}
\end{algorithm}
Define $V_n$ as the number of calls to QS, hence we have:\\
\[
V_0=1, V_1=3 
\]
\[
V_n=1+V_{k-1}+V_{n-k} \\
\]
Let a subset of $\left\lfloor{b}\right\rfloor$ values in $[n]$ be observed in all $n!$
permutations, then the number of possibilities is:\\
\[\binom{n}{b}(n-b)!\]
Because first, we choose $b$ terms from $n$, which is
$\binom{n}{b}$. And then the $(n-b)$ numbers are permutated.\\
Define $t$ as the number of terms on the right side not moved; $p$ as
the value of the pivot(is $p$ means this?).\\
Then we have:\\
\[\frac{1}{t!}\cdot\frac{1}{(n-p-t)!}\cdot\frac{1}{\binom{n-p}{t}}=\frac{1}{(n-p)!}\]\\
(I forgot what is the above function means) \\
Next, let's compute the value of $E[V_n]$:
\begin{align*}
E[V_n]&=v_n\\
&=1+\sum_{n=1}^nP[K=k](v_{k-1}+v_{n-k}) \\
&=1+\frac{1}{n}(\sum_{k=1}^nv_{k-1}+\sum_{k=1}^nv_{n-k}) \\
&=1+\frac{2}{n}\sum_{j=0}^{n-1}v_j \\
nv_n&=n+2\sum_{j=0}^{n-1}v_j \\
(n+1)v_{n+1}&=n+1+2\sum_{j=0}^nv_j \\
\text{Substract the above two equations, we get:}\\
(n+1)v_{n+1}-nv_n&=1+2v_n\\
(n+1)v_{n+1}&=1+(n+2)v_n \\
\frac{v_{n+1}}{n+2}&=\frac{1}{(n+1)(n+2)}+\frac{v_n}{n+1} \\
\end{align*}
Let $u_n=\frac{v_n}{n+1}$ \\
\begin{align*}
u_{n+1}&=\frac{1}{(n+1)(n+2)}+u_n \\
u_{n+1}&=\frac{3}{2}+\sum_{i=2}^{n+1}\frac{1}{i(i+1)} \\
&=\frac{3}{2}+\frac{1}{2}-\frac{1}{i+2} \\
&=2-\frac{1}{n+2}\\
\frac{v_{n+1}}{n+2}&=2-\frac{1}{n+2}\\
v_{n+1}&=2(n+2)-1\\
&=2n+3\\
v_n&=2(n-1)+3\\
&=2n+1\\
\text{Finally, we get:} \\
E[V_n]=2n+1\\
\end{align*}
Now, let's look at another version of the QS algorithm, which doesn't
compare the value of $i$ and $j$ at the beginning.\\
\begin{algorithm}[H]
QS($A,i,j$)\;
\If{$i<k-1$} {QS($A,i,k-1$)}
\If{$j>k-1$} {QS($A,k+1,j$)}
\caption{Another Version of the Quick Sort Algorithm}
\end{algorithm}
Agian, we calculate the expected value of $V_n$: Number of calls to
the QS function:\\
Same as before, assume $v_n=E[V_n]$:\\
First, we have:\\
\[
v_0=0\qquad v_1=0\qquad v_2=1 \\
\]
Same as before, we have: \\
\begin{align*}
\frac{v_{n+1}}{n+2}&=u_{n+1} \\
&=u_n+\frac{1}{(n+1)(n+2)} \\
&=u_2+\sum_{k=3}^{n+1}\frac{1}{k(k+1)}\\
&=u_2=\frac{1}{3}-\frac{1}{n+2}\\
\end{align*}
Because $u_2=\frac{v_2}{3}=\frac{1}{3}$, so \\
\begin{align*}
u_{n+1}&=\frac{2}{3}-\frac{1}{n+2} \\
&=\frac{1}{3}\cdot\frac{2n+1}{n+2} \\
\end{align*}
hence: \\
\[v_n=\frac{2n+1}{3}\]\\
Next, let's consider the number of term comparisons.\\
Define $C_n$ as the number of term comparisons. The term comparison
operations all occurs in the partition function. Define\\
\[
X_{ij}\twopartdef{1}{\text{$y_i$ and $y_j$ are compared}}{0}{\text{o.w.}}
\]
\\
Then we have:\\
\[
P[X_{ij}=1]=\frac{2}{j-i+1}\\
\]
Since at least on of $y_i$ or $y_j$ to be chosen as the pivot to make
them compared.\\ 
Then we have:\\
\[
C_n=\sum_{1\leq i<j\leq n}x_{ij}\\
\]
So:\\
\begin{align*}
E[C_n]&=\sum_{i+1}^n\sum_{j=i+1}^n\frac{2}{j-i+1}\\
\text{Let $k=j-i+1$}\\
&=\sum_{i=1}^n\sum_{k=2}^{n-i+1}\frac{2}{k}\\
&=\sum_{k=2}^{n-1}\sum_{i=1}^{n-k+1}\frac{2}{k}\\
&=\sum_{k=2}^{n-1}\frac{1}{k}\sum_{i=1}^{n-k+1}1\\
&=\sum_{k=2}^n\frac{n+1-k}{k}\\
&=(n+1)\sum_{k=2}^n\frac{1}{k}-\sum_{k+2}^n1\\
&=(n+1)(H_n-1)-n+1\\
&=2((n+1)H_n-2n)\\
&=2(n+1)H_n-4n\\
\end{align*}
Another Method:\\
\[\sum_{r=1}^nH_r=\sum_{r=1}^n\sum_{k=1}^r\frac{1}{k}=(n+1)H_n-n\]\\
Hence:\\
\begin{align*}
2\sum_{i=1}^n\sum_{k=2}^{n-i+1}&=2\sum_{i=1}^n(H_{n-i+1}-1)\\
&=-2n+\sum_{i=1}^nH_{n-i+1}\\
&=-2n+2\sum_{r=1}^nH_r\\
&=2(n+1)H_n-4n\\
\end{align*}
\end{document}

%%% Local Variables: 
%%% mode: latex
%%% TeX-master: t
%%% End: 
