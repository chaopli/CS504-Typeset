%        * * *    D E C L A R A T I O N S    * * *

% documentstyle.  (default point size 10pt. For Proceeding versions add the
%		   twocolumn option. If different margins are needed for 
%                  left/right pages then the option twoside can be used 
%                  together with the oddsidemargin/evensidemargin commands.
%		   For twocolumn option use the command columnsep to define
%		   the space between columns.

% page size parameters. (height & width & head size & leftmargin)
%
\documentclass[11pt,twoside]{article}
% \documentclass[11pt]{article}
\usepackage{times,mathptmx}
\usepackage{amsmath,amsthm,amssymb}
%\usepackage{latexsym}
%\usepackage{fp}
%\usepackage{mparhack}
\usepackage{stmaryrd}
\usepackage[dvips]{graphics}
\usepackage{epsfig}
\usepackage{ifthen}
%\usepackage{color}
\usepackage{eucal}
\usepackage[T1]{fontenc}
\usepackage{verbatim}
%\usepackage{comment}  --- only use locally in a file where you want it,
%since it interferes badly with the normal operation of the comment
%environment.
\usepackage{fancyvrb}
% \usepackage[notref,notcite]{showkeys}
\usepackage{curves}
\usepackage{epic}
\usepackage{eepic}
\usepackage{pstricks,pst-node,pst-text,pst-tree}
\usepackage{pstricks-add}
%\usepackage[verbose]{wrapfig}
%\usepackage{draftcopy}
%\usepackage{fancyhdr}
\usepackage[us]{datetime}
\usepackage{alltt}
\usepackage{xspace}
%\usepackage{multicol}
\setlength{\textwidth}{6.0in}
\setlength{\textheight}{8.0in}
\setlength{\oddsidemargin}{31pt}
\setlength{\evensidemargin}{0.35in}
\setlength{\headheight}{0.3in}
\setlength{\headsep}{0.5in}
\setlength{\columnsep}{1.0cm}
\vsize=8.5in
\hoffset=-0.4in
\voffset=-0.7in
\setlength{\parindent}{0pt}
\setlength{\parskip}{1ex}
%\renewcommand{\thefootnote}{\fnsymbol{footnote}} why???
\renewcommand{\thefootnote}{\arabic{footnote}}
%
% pagestyle - plain (page numbers in bottom and head is empty) or
%             empty (both bottom and head are empty)

%\pagestyle{fancy}
%\fancyhf{} % clears the slate
%\setlength{\headrulewidth}{0.4pt}
%\setlength{\footrulewidth}{0.0pt}
%\setlength{\plainfootrulewidth}{0.4pt}
%\setlength{\plainheadrulewidth}{0.0pt}
%\lhead{\fancyplain{}{ {\it Computer Science at
%{\small\wpi}:\ \ \ \ The Future (briefly)}} }
%
%\lhead{\fancyplain{}{\small {\it DRAFT DRAFT DRAFT DRAFT DRAFT 
%DRAFT DRAFT DRAFT DRAFT DRAFT DRAFT }}}
%\lfoot{\fancyplain{\small {\it DRAFT DRAFT DRAFT DRAFT DRAFT DRAFT DRAFT
%DRAFT DRAFT DRAFT DRAFT DRAF}}{\small {\it DRAFT DRAFT DRAFT DRAFT DRAFT
%DRAFT DRAFT DRAFT DRAFT DRAFT DRAFT DRAF}}}
%\chead{}
%\rhead{\fancyplain{}{\rm\thepage}}
%\cfoot{}
%\rfoot{}
%\thispagestyle{fancyplain}  % This is for the title page
% \begin{Verbatim}[commandchars=\\\{\},
%       codes={\catcode`$=3\catcode`^=7\catcode`_=8}]....need find others

%
%        * * *    M A C R O S    * * *
%
\newcommand{\refeq}[1]{Eq.\;(\ref{#1})}
\newcommand{\pp}{\paragraph}
\newcommand{\noi}{\noindent}
\newcommand{\wpi}{{\sc wpi}\xspace}
\newcommand{\cs}{{\sc cs}\xspace}
%\newcommand{\pmf}{{\sc pmf}\xspace}
\newcommand{\pmf}{ probability mass function\xspace}
\newcommand{\pgf}{{\sc pgf}\xspace}
%\newcommand{\pgf}{ probability generating function\xspace}
\newcommand{\ogf}{{\sc ogf}\xspace}
%\newcommand{\ogf}{ ordinary generating function\xspace}
\newcommand{\egf}{{\sc egf}\xspace}
\newcommand{\mgf}{ moment generating function\xspace}
%\newcommand{\egf}{ exponential generating function\xspace}
\newcommand{\pdf}{ probability density function\xspace}
\newcommand{\gf}{generating function\xspace}
% \newcommand{\gf}{{\sc gf}\xspace}
\newcommand{\bc}{binomial coefficient\xspace}
\newcommand{\pfd}{ partial fraction decomposition\xspace}
%\newcommand{\gf}{ generating function\xspace}
\newcommand{\lf}{\lfloor}
\newcommand{\rf}{\rfloor}
\newcommand{\lc}{\lceil}
\newcommand{\rc}{\rceil}
\newcommand{\lp}{\left(}
\newcommand{\rp}{\right)}
\newcommand{\lbr}{\left[}
\newcommand{\rbr}{\right]}
\newcommand{\ul}{\underline}
\newcommand{\bbb}{$\bullet$  }
\newcommand{\fns}{\footnotesize}
\newcommand{\DS}{\displaystyle}
\newcommand{\bu}{\bullet}
\newcommand{\eod}{\vrule height 6pt width 5pt depth 0pt}
\newcommand{\parsec}{\par\noindent}
\newcommand{\bigspace}{\bigskip\parsec}
\newcommand{\med}{\medskip\parsec}
%\newcommand{\ss}{\sigma}
\newcommand{\pr}{{\rm Pr}}
\renewcommand{\Pr}{\ensuremath{\mathbb{P}}\xspace}
\newcommand{\E}{\ensuremath{\mathbb{E}}\xspace}
\newcounter{hours}
\newcounter{minutes}
\newcommand{\ptime}{\setcounter{hours}{\time/60}%
     \setcounter{minutes}{\time-\value{hours}*60} \thehours:\theminutes}
\newcommand{\todo}[1]{{\rule{0.5em}{1ex}\textbf{ #1}}}
\renewcommand{\baselinestretch}{1.0}
\renewcommand{\baselinestretch}{1.2}
\newenvironment{singlespace}{\def\baselinestretch{1.0}\large\normalsize}{\par}
\newcommand{\nsection}[1]{\section[#1]{\hspace{-1em}.\hspace{1em}#1}}
%\newcommand{\mbf}[1]{\ensuremath{\text{\textbf{\emph{\;#1}}}}}
\newcommand{\mbf}[1]{\ensuremath{\text{\textbf{\emph{#1}}}}}
%\newcommand{\comment}[1]{}
\newcommand{\propsubset}
    {\stackrel{\subset}{\scriptscriptstyle \not {\scriptstyle =}}}
\newcommand{\hypoth}   {\stackrel{?}{=}}
\newcommand{\hypothge}   {\stackrel{?}{\ge}}
\newcommand{\hypothle}   {\stackrel{?}{\le}}
\newcommand{\hypothl}   {\stackrel{?}{<}}
\newcommand{\lset}{\left \{}
\newcommand{\rset}{\right \}}
\newcommand{\halfn}{{\left\lfloor{n \over 2}\right\rfloor}}
\newcommand{\unl}{\underline}
\newcommand{\ovl}{\overline}
\renewcommand{\ge}{\geqslant}
\renewcommand{\le}{\leqslant}
\renewcommand{\geq}{\geqslant}
\renewcommand{\leq}{\leqslant}
\newcommand{\EE}{{\cal E}}
\newcommand{\FF}{{\cal F}}
\newcommand{\eps}{\varepsilon}
\newcommand{\cA}{{\mathcal A}} \newcommand{\cB}{{\mathcal B}}
\newcommand{\cC}{{\mathcal C}} \newcommand{\cD}{{\mathcal D}}
\newcommand{\cE}{{\mathcal E}} \newcommand{\cF}{{\mathcal F}}
\newcommand{\cG}{{\mathcal G}} \newcommand{\cH}{{\mathcal H}}
\newcommand{\cI}{{\mathcal I}} \newcommand{\cJ}{{\mathcal J}}
\newcommand{\cK}{{\mathcal K}} \newcommand{\cL}{{\mathcal L}}
\newcommand{\cM}{{\mathcal M}} \newcommand{\cN}{{\mathcal N}}
\newcommand{\cO}{{\mathcal O}} \newcommand{\cP}{{\mathcal P}}
\newcommand{\cQ}{{\mathcal Q}} \newcommand{\cR}{{\mathcal R}}
\newcommand{\cS}{{\mathcal S}} \newcommand{\cT}{{\mathcal T}}
\newcommand{\cU}{{\mathcal U}} \newcommand{\cV}{{\mathcal V}}
\newcommand{\cW}{{\mathcal W}} \newcommand{\cX}{{\mathcal X}}
\newcommand{\cY}{{\mathcal Y}} \newcommand{\cZ}{{\mathcal Z}}
\newcommand{\CC}{\mathbb{C}} \newcommand{\NN}{\mathbb{N}}
\newcommand{\OO}{\mathbb{O}} \newcommand{\II}{\mathbb{I}}
\newcommand{\RR}{\mathbb{R}} \newcommand{\TT}{\mathbb{T}}
\newcommand{\ZZ}{\mathbb{Z}} \newcommand{\QQ}{\mathbb{Q}}

%\newcommand{\eqsp}{\mbox{\ \ \ \ \ }}
\newcommand{\grad}{\ensuremath{^{\circ}}} % the little circle for degrees
%\newcommand{\halfp}[1]{\frac{1}{2} \cdot \left ( #1 \right )}
\newcommand{\uff}{\leftrightarrow}% short single 2-way arrow
\newcommand{\ftz}{\footnotesize}
\newcommand{\frct}[2]
{\ensuremath{\raisebox{.3ex}{\ftz #1}\!/\!\raisebox{-0.3ex}{\ftz#2}}}
\newcommand{\frcts}[2]
{\ensuremath{\raisebox{.3ex}{\ftz #1}\!/\!\raisebox{-0.3ex}{\ftz#2}}\xspace}
%\newcommand{\half}{\ensuremath{\frac{1}{2}}}
\newcommand{\half}{\frct{1}{2}}
\newcommand{\bsl}{\ensuremath{\backslash}}
\newcommand{\eq}{\equiv}
\newcommand{\eqf}{\equiv_f}
\newcommand{\rv}{random variable\xspace}
\newcommand{\lar}{\Longrightarrow}  %double arrow (=implies)
%\newcommand{\implies}{\Longrightarrow}  %double arrow 
\newcommand{\olar}{\longrightarrow}  %single arrow
\newcommand{\loro}{\longleftrightarrow}  %single arrow
\newcommand{\Loro}{\Longleftrightarrow}  %double arrow
\newcommand{\laa}{\ensuremath{\leftarrow}}
\newcommand{\lbb}{\ensuremath{\rightarrow}}
% both of those below should be replaced by \to
%\newcommand{\oar}{\rightarrow}
%\newcommand{\con}{\rightarrow}
\newcommand{\cd}{\mbox{$\stackrel{d}{\rightarrow}$}}
\newcommand{\convd}{\mbox{$\stackrel{d}{\longrightarrow}$}}
\newcommand{\cp}{\mbox{$\stackrel{p}{\rightarrow}$}}
\newcommand{\map}{{\sc maple}\xspace}
\newcommand{\eqdef}{\stackrel{\rm def}{=}}
\newcommand{\abs}[1]{\left| #1\right|}
\newcommand{\set}[1]{\left\{ #1\right\}}
\newcommand{\Stwo}[2]{\left\langle{#1\atop #2}\right\rangle}
\newcommand{\suml}{\sum\limits}
\newcommand{\prodl}{\prod\limits}
\newcommand{\pile}[2]{\genfrac{}{}{0pt}{}{#1}{#2}}  % like troff pile
\newcommand{\st}[2]{\genfrac{[}{]}{0pt}{}{#1}{#2}}  % stirling I
\newcommand{\stt}[2]{\genfrac{\{}{\}}{0pt}{}{#1}{#2}}  %stirling II
\newcommand{\eul}[2]{\genfrac{\langle}{\rangle}{0pt}{}{#1}{#2}}  %euler
\newcommand{\head}{{\textsf{H}}\xspace}
\newcommand{\tail}{{\textsf{T}}\xspace}

\newcommand{\dfsa}{\textsc{dfs}\xspace}
\newcommand{\bfsa}{\textsc{bfs}\xspace}
\newcommand{\dfs}{depth-first search\xspace}
\newcommand{\bfs}{breadth-first search\xspace}
\newcommand{\rhs}{right-hand side\xspace}
\newcommand{\lhs}{left-hand side\xspace}
\newcommand{\rht}{right-hand subtree\xspace}
\newcommand{\lht}{left-hand subtree\xspace}
\newcommand{\pih}{{\hat p}_i}
\newcommand{\pbar}{\mbox{\textbf{\textit{p}}}}
\newcommand{\nbar}{\mbox{\textbf{\textit{n}}}}
\newcommand{\zbar}{\mbox{\textbf{\textit{z}}}}
\def\squarebox#1{\hbox to #1{\hfill\vbox to #1{\vfill}}}
\newcommand{\ed}{\mbox{$ \ \stackrel{d}{=}$ \ }}
\newcommand{\bbx}{\mbox{}\kern2ex\hfill\mbox{$\Box$}}
% \renewcommand{\qed}{ \mbox{}\kern1ex\hfill\rule{1.4ex}{1.4ex}}
\newtheorem{Proposition}{Proposition}
\newtheorem{Theorem}{Theorem}
\newtheorem{Lemma}{Lemma}
\newtheorem{THEOREM}{Theorem}
%\newenvironment{theorem}{\begin{THEOREM} \hspace{-.80em} {\bf :} \rm}%
% the above version was before I switched to amsthm
\newenvironment{theorem}{\begin{THEOREM}  \rm}%
                        {\end{THEOREM}}
\newtheorem{PROPERTY}{Property}
\newenvironment{property}{\begin{PROPERTY} \hspace{-.80em} {\bf :} \rm}%
                        {\end{PROPERTY}}
%\newtheorem{COROLLARY}[THEOREM]{Corollary}
\newtheorem{COROLLARY}{Corollary}
%\newenvironment{corollary}{\begin{COROLLARY} \hspace{-.80em} {\bf :} \rm}%
\newenvironment{corollary}{\begin{COROLLARY} \hspace{-.10em} }%
                          {\end{COROLLARY}}
\newtheorem{LEMMA}{Lemma}
\newenvironment{lemma}{\begin{LEMMA} \hspace{-.10em} \rm}%
                      {\end{LEMMA}}
\newtheorem{OBSERVATION}{Observation}
\newenvironment{observation}{\begin{OBSERVATION} \hspace{-.80em} {\bf :} \rm}%
                      {\end{OBSERVATION}}
\newtheorem{CLAIM}{Claim}
\newenvironment{claim}{\begin{CLAIM} \hspace{-.10em}  \rm}%
                      {\end{CLAIM}}
\newtheorem{EXAMPLE}{Example}
\newenvironment{example}{\begin{EXAMPLE} \hspace{-.30em} {\bf :} \rm}%
                      {\end{EXAMPLE}}
% \newenvironment{proof}{\noindent {\bf Proof:} \hspace{.677em}}%  cf. \qed
%                      {\hfill \qed\\}
\newenvironment{sketch}{\noindent {\bf Proof sketch:} \hspace{.677em}}%
                      {\hfill \qed\\}
\newtheorem{DEFINITION}{Definition}
\newenvironment{definition}{\begin{DEFINITION} \hspace{-.80em} {\bf :} \rm}%
                      {\end{DEFINITION}}
\newenvironment{df}{\begin{DEFINITION}  \rm}%
          {\hfill\protect$\lhd$\end{DEFINITION}} % changed by MH 5/11/08
\newtheorem{REMARK}{Remark}
\newenvironment{remark}{\begin{REMARK} \hspace{-.80em} {\bf :} \rm}%
                      {\end{REMARK}}
\newenvironment{pic}[1]%
{\input{#1}\begin{figure}\centerline{\box\graph}\vspace*{\bigskipamount}}%
{\end{figure}}
\newenvironment{thecomment}{\setbox0=\vbox\bgroup}{\egroup}

%
%        * * *    D O C U M E N T    * * *
%

%\special{!userdict begin /start-hook{true statusdict /setduplexmode get exec}def end}

%\begin{document}

%\usepackage{maple2e}  ...if needed, disable epsfig in head
%  The following switch distinguishes between homework (solu=0) and its
%  solution (solu=1).
\usepackage{dcolumn}
% \usepackage{stmaryrd}
\newcommand{\bst}{binary search tree\xspace}
\setlength{\leftmargin}{0.00in}
\setlength{\leftmargini}{0.00in}
\setlength{\leftmarginii}{0.00in}
\newcommand{\hwno}{6}
\newcommand{\solu}{1}
\ifthenelse{\equal{\solu}{1}}%
{\renewenvironment{comment}{\  \\ {\bf Solution:\ \ } }}{}%

\begin{document}
\pagestyle{myheadings}
%\newcommand{\pfd}{partial fraction decomposition\xspace}
\ifthenelse{\equal{\solu}{0}}%
{\markboth{{\sc cs504:}{\it \ \ Homework \hwno}}{{\it \ \ Homework \hwno}}}%
{\markboth{{\sc cs504:}{\it \ \ Homework \hwno:  Solutions }}{{\it \ \ %
Homework \hwno:  Solutions}}}

\thispagestyle{empty}
\vspace*{-1.3in}
\begin{center}
\Large{CS504: Analysis of Computations and Systems --- Spring \the\year}
\end{center}

%\voffset=-0.7in
%\hoffset=-0.5in
%\null\vskip 1.1cm
\begin{center}
\ifthenelse{\equal{\solu}{0}}%
{{\large\bf Homework \hwno }\\[1.5ex]}
{{\bf\large \hspace*{0.30in}Solution for Homework \hwno}
\hspace*{0.70in}Posted: \currenttime,\  \today\\[1.5ex]}
\end{center}
\ifthenelse{\equal{\solu}{0}}%
{\hfill Due: Beginning of class, Wednesday, February 26, 2014\\}{}
%
%\begin{center}
%(I shall complete it within a few days)
%\end{center}

% \vspace*{-0.3in}

\begin{enumerate}
\item
In the following $f()$ and $g()$ are the \gf{s} of some
sequences, and therefore have power series developments. The symbols
$z$ and $a$ there are an indeterminate and a constant, respectively.
Explain the relations:\\
$\DS g(z)=f(z-a) ~~~\lar~~(1)~[(z+a)^n]\,f(z)=g_n;
\quad (2)~~ [ z^n ]\,f(z) = \sum_{k} \,\binom{k}{ n} g_k \,a^{k-n} $.\\
This can be used to connect the developments of the same function in two
points.
\begin{comment}
Evidently, if $g(z)=f(z-a)$ holds, presumably for any value of $z$,
then also $g(z+a)=f(z)$.  The \lhs has the expansion
$g(z+a)=\suml_{k\ge0}g_k(z+a)^k$.  Hence the first claim:
$\DS[(z+a)^n]g(z+a) =[(z+a)^n]\suml_{k\ge0}g_k(z+a)^k=g_n$.
And since $g(z+a)=f(z)$, the claim holds.

The second one is more direct:
\[
[z^n]f(z) = [z^n]g(z+a)
=[z^n]\suml_{k\ge0}g_k(z+a)^k
=\suml_{k\ge0}g_k[z^n] \sum_j \binom kj z^ja^{k-j}
=\suml_{k\ge0}g_k \binom kn a^{k-n}
\]
where the extraction operator annihilates all the terms with powers of $z$
which are not $n$.

{\em Now that I am done grading, two comments:}\\
1.  Note the ``annihilation'' or selection action of the operator
$[z^n]$.  It is often useful in simplifying complex expressions, as it
removes one summation level, by selecting a single term out of the summed
over sequence.\\
2.  A note of different kind; a couple of students did more or less what I
write above, but for the binomial expansion used the dummy index $n$ (in
lieu of my $j$), not
noticing that it is already in use, in the extraction operator.
This produced a meaningless mess.  Dummy indices must be ``fresh,'' and
newly-minted. 
\end{comment}
\item
Expand the \ogf $d(x) = (4-2x+x^2)^{-1}$.  Provide for $d_n$ a closed form
expression, without complex numbers.
\begin{comment}
%\mbf{(b)\ \ }
Complex numbers enter the calculation here since the given quadratic does
not have real roots; instead, they are $1\pm i\sqrt3$.  We find
\begin{align*}
d_n &= [x^n]\,\frac1{4-2x+x^2} = [x^n]\frac1{(1-i\sqrt{3} -x)(1+i\sqrt{3}
-x)} =[x^n] \frac i{2\sqrt3}\lp \frac1{1+i\sqrt{3} -x} -\frac1{1-i\sqrt{3}
-x}\rp,
\end{align*}
following a \pfd.  This can be further developed,
\begin{align*}
d_n &= \frac i{2\sqrt3} [x^n] \lp
\frac1{1+i\sqrt3} \frac1{1 -\frac x{1+i\sqrt{3}}}
- \frac1{1-i\sqrt3} \frac1{1 -\frac x{1-i\sqrt{3}}} \rp\\
 &= \frac i{2\sqrt3} \lp
\frac1{1+i\sqrt3} \lp\frac1{1+i\sqrt3} \rp^n
-\frac1{1-i\sqrt3} \lp\frac1{1-i\sqrt3} \rp^n \rp.
\end{align*}
While this involves complex numbers, they are complex conjugates, and the
result of the subtraction is real. The expression can be somewhat
simplified; however, you were asked for an
explicitly real expression, not just real-valued, so we need to continue
beyond a simplification.\\
Since $(1-i\sqrt3) (1+i\sqrt3)=4$, we can write 
$\DS\frac1{1-i\sqrt3} = \frac14\,(1+i\sqrt3)$, and
\[
d_n = \frac i{2\sqrt3}\frac1{4^{n+1}}\lp (1-i\sqrt3)^{n+1}-
(1+i\sqrt3)^{n+1}\rp.
\]
We continue with the polar representation of the complex numbers:
$u+iv=r\,\exp\lp i\tan^{-1} (v/u)\rp$, where $r$ is the absolute value of
the number, $\sqrt{u^2+v^2}$, which is 2 for the numbers we have, and the
tangent of the angle, $\theta$ in the diagram is $\sqrt3$, hence
$\theta=\pi/3$.

\psset{unit=2.2mm,arrows=-,arrowsize=4pt,nodesep=0pt,linewidth=0.6pt}
\begin{pspicture}(-20,-20)(15,20)
% \pscircle[linecolor=blue](25,0){20}
\psline[arrows=->](0,0)(12,0)
\psline[arrows=->](0,-19)(0,19)
\pnode(0,0){orig} \cnode*(10,17.32){1pt}{b} \cnode*(10,-17.32){1pt}{c}
\ncline{orig}{b}\naput[nrot=:U]{$r=2$}
\ncline{orig}{c} \ncline{b}{c}
\rput(14.0, 0){$\Re$}
\rput(-2.0, 17){$\Im$}
\rput(2.6,2.0){$\theta$}
\rput(13.5,17.1){$1+i\sqrt3$}
\rput(13.5,-17.1){$1-i\sqrt3$}
\end{pspicture}

Therefore $1\pm i\sqrt3 = 2e^{\pm i\theta}$, and
$(1\pm i\sqrt3)^{n+1} = 2^{n+1}e^{\pm i(n+1)\pi/3}$.  We use de Moivre
formula, the facts that $\cos(-x)= \cos(x)$ while  $\sin (-x) = -\sin (x)$,
and substitute into $d_n$
\[
d_n = \frac i{2\sqrt3}\frac{2^{n+1}}{4^{n+1}}
\lp \cos (n+1)\frac\pi3 -i \sin (n+1)\frac\pi3 -\cos (n+1)\frac\pi3
-i\sin (n+1)\frac\pi3\rp
\]
The cosines cancel, the sines add, multiplying $i$ by itself corrects the
sign, hence,
\[
d_n= \frac1{2^{n+1}\,\sqrt3}\sin  (n+1)\frac\pi3,
\qquad \text{ while}\quad 
 \sin k\frac\pi3 = \begin{cases} 0& k=3r\\ 
 \frac{\sqrt3}2 & k=6r+1, \quad\text{ or } 6r+2\\
 -\frac{\sqrt3}2 & k=6r-1, \quad\text{ or } 6r-2,\end{cases}
\]
for any integer $r$. Therefore 
\[
d_n=\frac{V_n}{2^{n+2}}, \qquad \text{ where }\quad 
V_n = \begin{cases} 0, & n=3r-1\\ 1, & n=6r, \quad\text{ or }~6r+1\\
 -1,  & n=6r-2, \quad\text{ or }~6r-3. \end{cases}
\]
And then it is always a smart idea to check, and the easiest way is to ask
a computer with an algebra package (I use \map) to do it.  Here it is on
\map, enough to convince me I made no error:
\begin{verbatim}
t := 1/(4-2*x+x^2): series(t, x=0, 12);
1   1     1   3   1   4    1   6    1   7    1    9    1    10   / 12\
- + - x - -- x  - -- x  + --- x  + --- x  - ---- x  - ---- x   +O\x  /
4   8     32      64      256      512      2048      4096  
\end{verbatim}

\end{comment}
\end{enumerate}
\end{document}
\item \label{item4}
\begin{comment}
\end{comment}


\end{enumerate}
\end{document}
\item
\begin{comment}
\end{comment}
\begin{align*}
\end{align*}
\bigskip
\item
\begin{comment}
\end{comment}

\noindent
{\bf Part II.}

Do the following exercises from Heck: {\em Introduction to Maple},
pp63--4:\\
2,\ \ \ 
3(a),(d)\ \ \ 
6,\ \ \ 
11,\ \ \ 
14\\
\noindent
{\bf 3.}
%\rule{6in}{1mm}
\end{document}
