\documentclass[11pt]{article}
\usepackage{times,mathptmx}
\usepackage{amsmath,amsthm,amssymb}
\usepackage{xspace}
\newcommand{\DS}{\displaystyle}
\newcommand{\pmfa}{{\sc pmf}\xspace}
\newcommand{\pmf}{ probability mass function\xspace}
\newcommand{\pgfa}{{\sc pgf}\xspace}
\newcommand{\pgf}{ probability generating function\xspace}
\newcommand{\ogfa}{{\sc ogf}\xspace}
\newcommand{\ogf}{ ordinary generating function\xspace}
\newcommand{\egfa}{{\sc egf}\xspace}
\newcommand{\mgf}{ moment generating function\xspace}
\newcommand{\egf}{ exponential generating function\xspace}
\newcommand{\pdf}{ probability density function\xspace}
\newcommand{\gf}{generating function\xspace}
\newcommand{\gfa}{{\sc gf}\xspace}
\newcommand{\bc}{binomial coefficient\xspace}
\newcommand{\pfd}{ partial fraction decomposition\xspace}
\newcommand{\lhs}{left-hand side\xspace}
\newcommand{\rhs}{right-hand side\xspace}
\newcommand{\lf}{\lfloor}
\newcommand{\rf}{\rfloor}
\newcommand{\lc}{\lceil}
\newcommand{\rc}{\rceil}
\newcommand{\lp}{\left(}
\newcommand{\rp}{\right)}
\newcommand{\lbr}{\left[}
\newcommand{\rbr}{\right]}
\newcommand{\ul}{\underline}
\newcommand{\ovl}{\overline}
\newcommand{\RR}{\mathbb{R}} \newcommand{\TT}{\mathbb{T}}
\newcommand{\ZZ}{\mathbb{Z}} \newcommand{\QQ}{\mathbb{Q}}
\newcommand{\cA}{{\mathcal A}} \newcommand{\cB}{{\mathcal B}}
\newcommand{\cC}{{\mathcal C}} \newcommand{\cD}{{\mathcal D}}
\setlength{\parindent}{0pt}
\setlength{\parskip}{1ex}
\newtheorem{DEFINITION}{Definition}
\newenvironment{df}{\begin{DEFINITION} \hspace{.30em}  \rm}%
          {\hfill\protect$\lhd$\end{DEFINITION}}
\newtheorem{EXAMPLE}{Example}
\newenvironment{example}{\begin{EXAMPLE} \hspace{.30em} \rm}%
                      {\end{EXAMPLE}}
\author{Prof. Micha Hofri/Typeset: Chao Li}
\title{Generating Functions}
\usepackage{amsmath}
\begin{document}
\maketitle
\section*{Generating Functions}
\begin{description}
\item[Definition] \hfill \\
Assume we have a number serie \(\{a_n\}_{n\geq 0}\), then its Ordinary
Generating Function(OGF) is \(a(x)=\sum_{n\geq 0}a_nx^n\); its Exponential
Generating Function(EGF) is \(\hat{a}(x)=\sum_{n\geq 0}a_n\frac{x^n}{n!}\).
\end{description}
Now let's look at a simple example: let \(n_n=a_{n+1}, n\geq 0\),
first on the OGF, then \\
\[
b(x)=\sum_{n\geq 0}b_nx^n=\frac{1}{x}\sum_{n\geq 0}a_{n+1}x^{n+1} =\frac{a(x)-a_0}{x}
\]
Next on the EGF, then\\
\begin{align*}
\hat{b}(x)&=\sum_nb_n\frac{x^n}{n!} \\
&=\sum_{n\geq 0}a_{n+1}\frac{x^n}{n!} \\
&=\sum_{n\geq 1}a_{n+1}\frac{(n+1)x^{n+1}}{x(n+1)!} \\
&=\sum_{n\geq 1}a_{n+1}\frac{(x^{n+1})'}{(n+1)!} \\
&=D_x\sum_{n\geq 0}a_{n+1}\frac{x^{n+1}}{(n+1)!}
\end{align*}
\section*{Convolution}
Assume we have two number series \(a:\{a_n\}_{n\geq 0},b:\{b_n\}_{n\geq 0}\) \\
Define
\[
c=a*b
\]
then
\begin{align*}
c_n &= \sum_{k}^{}a_kb_{n-k} \\
	&= \sum_{j}^{}b_ja_{n-j}
\end{align*}
\begin{align*}
c(x) &= \sum_{n\geq 0}c_nx^n	\\
	 &= \sum_n\sum_ka_kb_{n-k}x^n \\
	 &= \sum_{k\geq 0}a_k\sum_{n\geq k}b_{n-k}x^n \\
	 &=\underbrace{\sum_{k\geq 0}a_kx^k}_{a(x)}\underbrace{\sum_{j\geq 0}b_jx^j}_{b(x)}
\end{align*}
So we get \\
\[c(x)=a(x)b(x)\]
and 
\[
\hat{c}(x)=\sum_nc_n\frac{x^n}{n!}=\sum_{n\geq 0}\sum_ka_kb_{n-k}\frac{x^n}{n!}
\]
Next, let us define 
\[
r_n=\sum_{k}\binom{n}{k}a_kb_{n-k}
\]
then we have\\
\[
\hat{r}(x)=\sum_n\sum_k\binom{n}{k}a_kb_{n-k}\frac{x^n}{n!}=\sum_ka_k\frac{x^k}{k!}\sum_{n\geq k}b_{n-k}\frac{x^{n-k}}{(n-k)!}
\]
Let's look at an example based on this definition\\
Let \(a_n=1,b_n=f^n\), and \(u_n=\sum_{k=0}^n1\cdot f^k\binom{n}{k}\), which matches the above definition\\
then we have the Exponential Generating Function of all the three terms \\
\[
\hat{a}(x)=\sum1\cdot \frac{x^n}{n!}=e^x
\]
\[
\hat{b}(x)=\sum\frac{(fx)^n}{n!}=e^{fx}
\]
\[
\hat{u}(x)=e^xe^{fx}=e^{x(1+f)}=\sum u_n\frac{x^n}{n!}=\sum_n(1+f)^n\frac{x^n}{n!}
\]
Then, from the formula of Binomial, we have\\
\[
u_n=(1+f)^n
\]
Next let's calculate the generating function of the Harmonic Number \\
\[
H_n=\sum_{j=1}^n\frac{1}{j}
\]
Then the OGF of it is \\
\begin{align*}
H(x) &= \sum_{n\geq 1}H_nx^n	\\
	 &= \sum_{n\geq 1}\sum_{j\geq 1}\frac{x^n}{j} \\
	 &= \sum_{j\geq 1}\frac{1}{j}\underbrace{\sum_{n\geq j}x^n}_{\frac{x^j}{1-x}} \\
	 &= \frac{1}{1-x}\sum_{j\geq 1}\frac{x^j}{j} \\
	 &= \frac{1}{1-x}ln(1-x)
\end{align*}
\end{document}
%%% Local Variables: 
%%% mode: latex
%%% TeX-master: t
%%% End: 
