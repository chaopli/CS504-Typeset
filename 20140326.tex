\documentclass[11pt]{article}
\usepackage{times,mathptmx}
\usepackage{amsmath,amsthm,amssymb}
\usepackage{xspace}
\newcommand{\DS}{\displaystyle}
\newcommand{\pmfa}{{\sc pmf}\xspace}
\newcommand{\pmf}{ probability mass function\xspace}
\newcommand{\pgfa}{{\sc pgf}\xspace}
\newcommand{\pgf}{ probability generating function\xspace}
\newcommand{\ogfa}{{\sc ogf}\xspace}
\newcommand{\ogf}{ ordinary generating function\xspace}
\newcommand{\egfa}{{\sc egf}\xspace}
\newcommand{\mgf}{ moment generating function\xspace}
\newcommand{\egf}{ exponential generating function\xspace}
\newcommand{\pdf}{ probability density function\xspace}
\newcommand{\gf}{generating function\xspace}
\newcommand{\gfa}{{\sc gf}\xspace}
\newcommand{\bc}{binomial coefficient\xspace}
\newcommand{\pfd}{ partial fraction decomposition\xspace}
\newcommand{\lhs}{left-hand side\xspace}
\newcommand{\rhs}{right-hand side\xspace}
\newcommand{\lf}{\lfloor}
\newcommand{\rf}{\rfloor}
\newcommand{\lc}{\lceil}
\newcommand{\rc}{\rceil}
\newcommand{\lp}{\left(}
\newcommand{\rp}{\right)}
\newcommand{\lbr}{\left[}
\newcommand{\rbr}{\right]}
\newcommand{\ul}{\underline}
\newcommand{\ovl}{\overline}
\newcommand{\RR}{\mathbb{R}} \newcommand{\TT}{\mathbb{T}}
\newcommand{\ZZ}{\mathbb{Z}} \newcommand{\QQ}{\mathbb{Q}}
\newcommand{\cA}{{\mathcal A}} \newcommand{\cB}{{\mathcal B}}
\newcommand{\cC}{{\mathcal C}} \newcommand{\cD}{{\mathcal D}}
\setlength{\parindent}{0pt}
\setlength{\parskip}{1ex}
\newtheorem{DEFINITION}{Definition}
\newenvironment{df}{\begin{DEFINITION} \hspace{.30em}  \rm}%
          {\hfill\protect$\lhd$\end{DEFINITION}}
\newtheorem{EXAMPLE}{Example}
\newenvironment{example}{\begin{EXAMPLE} \hspace{.30em} \rm}%
                      {\end{EXAMPLE}}
\author{Prof. Micha Hofri/Typeset:  Chao Li}
\title{Binomial Coefficients}

\begin{document}
\maketitle
\section*{Binomial Coefficients}
First, let's look at some formulations \\
\[
\binom{n}{k}=\frac{n!}{k!(n-k)!}
\]
Stirling Approximation \\
\begin{align*}
n!&=n^n\sqrt{2\pi n}e^{-n}\left(1+\frac{1}{12n}+O\left(\frac{1}{n^2}\right)\right) \\
&=\frac{n^ne^{-n}\sqrt{2\pi n}}{k^ke^{-k}\sqrt{2\pi k}(n-k)^{n-k}e^{-n+k}\sqrt{2\pi (n-k)}} \\
&=\left(\frac{n}{k}\right)^k\left(\frac{n}{n-k}\right)^{n-k}\sqrt{\frac{n}{k(n-k)2\pi}}
\end{align*}
Next, we have seen before
\[
\binom{n}{k+1}=\frac{n!}{(k+1)!(n-k-1)!}=\frac{n!}{k!(n-k)!}\cdot \frac{n-k}{k+1}=\binom{n}{k}\frac{n-k}{k+1}
\]
\[
\binom{a}{b}=\frac{a^{\underline{b}}}{b!}=\frac{a(a-1)\cdots (a-b+1)}{b!}
\]
in which \(a\) is any number and \(b\in N\), otherwise \(\binom{a}{b}=0\).\\
Assume we have \(n,s\gg 1\) and \(t\ll n,s\), saying \(t^2\in o(S, n-s)\) and \(t\in Z\).
Then
\begin{align*}
\binom{n}{s+t}&=\frac{n^{\underline{s+t}}}{(s+t)!} \\
			  &=\frac{n^{\underline{s}}(n-s)^{\underline{t}}}{s!(s+t)^{\underline{t}}} \\
			  &=\binom{n}{s}\frac{(n-s)(n-s-1)(n-s-2)\cdots (n-s-t+1)}{(s+t)(s+t-1)\cdots (s+t-t+1)} \\
			  &=\binom{n}{s}\frac{(n-s)^t}{s^t}\underbrace{\frac{1\cdot (1-\cfrac{1}{n-s})(1-\cfrac{2}{n-s})\cdots (1-\cfrac{t-1}{n-s})}{(1+\cfrac{1}{s})(1+\cfrac{2}{s})\cdots (1+\cfrac{t}{s})}}_D
\end{align*}
As \((1+\frac{1}{s})(1+\frac{2}{s})\cdots (1+\frac{t}{s})=\frac{t(t+1)(3t^2-13t-2)}{24}\) \\
So 
\begin{align*}
D&=\cfrac{1-\cfrac{t(t-1)}{2(n-2)}+\cfrac{t(t-1)(t-2)(3t-1)}{24(n-s)^2}}{1+\cfrac{t(t+1)}{2s}+\cfrac{t(t+1)(3t^2-13t-2)}{24s^2}} \\
 &=\left(1-\frac{t(t-1)}{2(n-s)}\right)\left(1+\frac{t(t+1)}{2s}\right)+o\left(\frac{t^2}{s^2}\right) \\
 &=1-\frac{t(t+1)}{2s}-\frac{t(t-1)}{2(n-s)}+o\left(t^2\left(\frac{1}{n-s}+\frac{1}{s}\right)\right)
\end{align*}
Finally, we got
\[
\binom{n}{s+t}=\binom{n}{s}\left(\frac{n-s}{s}\right)^t\left(1-\frac{t(t+1)}{2s}-\frac{t(t-1)}{2(n-s)}+o\left(t^2\left(\frac{1}{s}+\frac{1}{n-s}\right)\right)\right)
\]
\section*{Relative Error}
Assume we have \(D_n=n!\sum_{k=0}^n\frac{(-1)^k}{k!}\) then, we have
\begin{align*}
D_n &= n!\left(\frac{1}{e}-\sum_{k>n}\frac{(-1)^k}{k!}\right)+O\left(\frac{1}{(n+1)!}\right) 
	&= \frac{n!}{e}-O(\frac{1}{n})
\end{align*}
Then \(O(1)\) is called the relative error, and \(\left(1+O(1)\right)\) is called relative error factor \\


\end{document}
 
%%% Local Variables: 
%%% mode: latex
%%% TeX-master: t
%%% End: 
